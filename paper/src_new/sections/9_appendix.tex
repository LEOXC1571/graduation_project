\cleardoublepage
{
\chapternonum{附录}

\appendixsecmajornumbering

% \section{负对数似然估计}

% \begin{algorithm}[H]
%     \caption{Log-likelihood estimator for EDMs}
%     \label{alg:likelihood_edm}
%     \begin{algorithmic}
%     \STATE {\bfseries Input:} Data point $x$, neural network $\phi$
%     \STATE Sample $t \sim \mathcal{U}(1, \ldots, T)$, $\varepsilon_t \sim \mathcal{N}(\mathbf{0}, I)$, subtract center of gravity from $\varepsilon^{(x)}_t$ in $\varepsilon_t = [\varepsilon^{(x)}_t, ~\varepsilon^{(h)}_t$]
%     \STATE $z_t = \alpha_t [x, h] + \sigma_t \varepsilon_t$
%     \STATE $\mathcal{L}_t = \frac{1}{2}(1 - \mathrm{SNR}(t-1) / \mathrm{SNR}(t)) ||\varepsilon_t - \phi(z_t, t)||^2$
%     \STATE Sample $\varepsilon_0 \sim \mathcal{N}(\mathbf{0}, I)$, subtract center of gravity from $\varepsilon^{(x)}_0$ in $\varepsilon_0 = [\varepsilon^{(x)}_0, \varepsilon^{(h)}_0]$
%     \STATE $z_0 = \alpha_0 [x, h] + \sigma_0 \varepsilon_0$
%     \STATE $\mathcal{L}_0 = \mathcal{L}_0^{(x)} + \mathcal{L}_0^{(h)} = -\frac{1}{2} ||\varepsilon - \phi(z_0, 0)||^2 - \log Z + \log p(h | z_0^{(h)})$
%     \STATE $\mathcal{L}_{\text{base}} = -\mathrm{KL}(q(z_T | x, h) | p(z_T)) = -\mathrm{KL}(\mathcal{N}_{xh}(\alpha_T [x, h], \sigma_T^2 I) | \mathcal{N}_{xh}(\mathbf{0}, I))$
%     \STATE Return $\hat{\mathcal{L}} = T \cdot \mathcal{L}_t + \mathcal{L}_0 + \mathcal{L}_{\text{base}}$
%     \end{algorithmic}
% \end{algorithm}

\section{化学键长}
本文在生成药物分子中的化学键时,依据计算化学领域常见的原子键长作为参考。表~\ref{tab:single_bond_typ_dist}~,表~\ref{tab:double_bond_typ_dist}~和表~\ref{tab:tri_bond_typ_dist}~列出了出现在GEOM-QM9和GEOM-Drugs中原子类型间所有可能的化学键的典型键长。表中数值代表对应化学键的典型长度(皮米),而横线代表该两种原子间不可能存在稳定的对应化学键连接,原子间距离短于典型键长则可以被认为被对应的键连接。在实际计算中,对单键,双键和三键典型键长分别有10,5和3皮米的冗余。例如,对于两个距离136皮米的碳原子,他们之间距离虽然大于134皮米,小于154皮米,但本文认定136皮米小于134+5皮米,则认为这两个碳原子由双键连接。

\begin{table}[!h]
    \footnotesize
    \centering
    \caption{典型单键键长}
    \label{tab:single_bond_typ_dist}
    \begin{tabular}{l | r r r r r r r r r r r r r r r r r}
    \toprule
    & H & C & O & N & P & S & F & Si & Cl & Br & I & B & As \\ \midrule
    H & 74 & 109 & 96 & 101 & 144 & 134 & 92 & 148 & 127 & 141 & 161 & 119 & 152 \\
    C & 109 & 154 & 143 & 147 & 184 & 182 & 135 & 185 & 177 & 194 & 214 & - & - \\
    O & 96 & 143 & 148 & 140 & 163 & 151 & 142 & 163 & 164 & 172 & 194 & - & - \\
    N & 101 & 147 & 140 & 145 & 177 & 168 & 136 & - & 175 & 214 & 222 & - & - \\
    P & 144 & 184 & 163 & 177 & 221 & 210 & 156 & - & 203 & 222 & - & - & - \\
    S & 134 & 182 & 151 & 168 & 210 & 204 & 158 & 200 & 207 & 225 & 234 & - & - \\
    F & 92 & 135 & 142 & 136 & 156 & 158 & 142 & 160 & 166 & 178 & 187 & - & - \\
    Si & 148 & 185 & 163 & - & - & 200 & 160 & 233 & 202 & 215 & 243 & - & - \\
    Cl & 127 & 177 & 164 & 175 & 203 & 207 & 166 & 202 & 199 & 214 & - & 175 & - \\
    Br & 141 & 194 & 172 & 214 & 222 & 225 & 178 & 215 & 214 & 228 & - & - & - \\
    I & 161 & 214 & 194 & 222 & - & 234 & 187 & 243 & - & - & 266 & - & - \\
    B & 119 & - & - & - & - & - & - & - & 175 & - & - & - & - \\
    As & 152 & - & - & - & - & - & - & - & - & - & - & - & - \\
    \bottomrule
    \end{tabular}
\end{table}
    
\begin{table}[H]
    \centering
    \begin{minipage}[t]{.5\textwidth}
    \begin{table}[H]
    \footnotesize
        \centering
        \caption{典型双键键长}
        \label{tab:double_bond_typ_dist}
        \begin{tabular}{l | r r r r r r r r r r r r r r r r r}
        \toprule
        & C & O & N & P & S \\ \midrule
        C & 134 & 120 & 129 & - & 160 \\
        O & 120 & 121 & 121 & 150 & - \\
        N & 129 & 121 & 125 & - & - \\
        P & - & 150 & - & - & 186 \\
        S & - & - & - & 186 & - \\
        \bottomrule
        \end{tabular}
    \end{table}
    \end{minipage}
    \begin{minipage}[t]{.4\textwidth}
    \begin{table}[H]
    \footnotesize
        \centering
        \caption{典型三键键长}
        \label{tab:tri_bond_typ_dist}
        \begin{tabular}{l | r r r r r r r r r r r r r r r r r}
        \toprule
        & C & O & N \\ \midrule
        C & 120 & 113 & 116 \\
        O & 113 & - & - \\
        N & 116 & - & 110 \\
        \bottomrule
        \end{tabular}
    \end{table}
    \end{minipage}
\end{table}

    % \begin{figure}[htb]
    %     \centering
    %     \includegraphics{example-image}
    %     \caption{附录中的图片}
    %     \label{fig:test-appendix}
    % \end{figure}
    % \clearpage
    
\section{超参数设置}
由于完整跑完一轮训练所需时间较长,本文未对超参数设置进行完整的实验。但经过初步实验,本文确定了实验的相关超参数。在QM9和Drugs数据集上的模型训练时,学习率设定为0.001,DTN层数设定为5层,特征维度设定为256,注意力头数设定为8,神经元遗忘率为0.1,独热编码原子类型标准化系数为0.25,原子序数标准化系数为0.1,化合价标准化系数为0.1。在QM9数据集上的模型训练中,扩散步数为500,每批样本数量设定为32,而在Drugs上的训练中,扩散步数为1000,每批样本数量为1,这一设定受制于GPU显存容量限制。为实现等效训练的效果,本文通过梯度累积的方式,实现了等效批样本数量64的训练效果。


% \section{难度可控的多项选择题生成实例}
}