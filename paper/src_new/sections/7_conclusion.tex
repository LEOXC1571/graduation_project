\chapter{结论与展望}
\label{chap:conclusion}

\section{结论}
本文提出了一种新的分子生成方法GFMDiff,它利用几何信息来帮助模型训练和图形形成。与早期的方法不同,GFMDiff充分利用空间信息来辅助表示学习,并促进准确的边生成,而不是严重依赖预定义规则来预测化学键的存在。我们采用了DTN作为去噪核,它是一种基于全局变换器的双跟踪神经网络,可以充分利用原子之间的配对距离和三元角度。同时引入了GFLoss来在每个采样步骤中积极干预化学键的形成。我们进行了全面的实验,评估了所提出技术相对于SOTA方法的有效性和性能优势。结果显示,GFMDiff能够生成具有准确构象的有效分子。鉴于在大分子基准测试中的表现,生成新的大样本仍需要进一步探索。

\section{展望}