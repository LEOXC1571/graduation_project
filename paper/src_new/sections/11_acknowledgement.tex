\cleardoublepage
\addcontentsline{toc}{chapter}{致谢}
\text{\zihao{3} \heiti 致谢}

日月如梭,文籍如海,探讨不及,朱黄敢怠。

硕士研究生阶段的时光如手中沙,不知不觉已经全部散入风中。在写下这段文字时,无数记忆碎片从我的脑海中闪过。有初入校园时初秋燥热的太阳下,我好奇的打量着新校园的一切;有冬日夜晚的冷风中,我骑车穿行在繁华的街边,心中惦念着文章的进度;有春日的傍晚,在学校的公园里和西湖边感受微风拂面;有夏天闷热中,充实的实习时光,夜晚回家时路过的水果摊卤味店的生活气息。如今经过两个四季轮回,我的手再次放在了命运之门的把手上,在对未来充满憧憬的同时,也有不少对硕士时光的感触。

物有本末,事有始终,知所先后,则近道矣。

首先要感谢的是王伟刚教授在硕士研究生期间对我的培养。在课程与科研上,王老师是我的领路人,悉心帮助我在科研路上迈出坚实的步伐。在研究遇到瓶颈时,王老师总是能够根据雄厚的知识储备给出独到的见解。在生活和与业规划中,王老师总是能从学生的角度出发考虑问题,在我遇到选择或是困难时,王老师也总是能根据丰富的阅历指引未来的方向。能够接受王老师的指导,是一件幸运的事。

其次,我要感谢我的父母,他们对我在生活,学习等方面的关心无微不至。在生活层面,他们十分关心我的衣食住行,时常也给我带来家的味道以缓解思乡之情。在学习和个人层面,他们也总对我论文写作,实习经历等关心有加。在我迷茫或是遇到挫折时,他们也总是给我坚定的支持与鼓励。

而后,我要感谢史妍,李超,姚晓婉,秦永菲,刘宇玮,王季欣,郭欣宇,朱黎彬等同学们。

最后,我要感谢张楚同学。


志之所趋,无远弗届,穷山距海,不能限也。

\newpage