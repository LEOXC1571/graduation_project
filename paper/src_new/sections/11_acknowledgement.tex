\cleardoublepage
\addcontentsline{toc}{chapter}{致谢}
\text{\zihao{3} \heiti 致谢}

日月如梭,文籍如海,探讨不及,朱黄敢怠。硕士研究生阶段的时光如手中沙,不知不觉已经全部散入风中。在写下这段文字时,无数记忆碎片从我的脑海中闪过。如今经过两个四季轮回,我的脚步需要再次跨过又一个人生之门,在对未来充满憧憬的同时,也有不少对硕士时光的感触。

物有本末,事有始终,知所先后,则近道矣。首先要感谢的是王伟刚教授在硕士研究生期间对我的培养。在课程与科研上,王老师是我的领路人,悉心帮助我在科研路上迈出坚实的步伐。在研究遇到瓶颈时,王老师总是能够根据雄厚的知识储备给出独到的见解。在生活和与业规划中,王老师总是能从学生的角度出发考虑问题。能够接受王老师的指导,是一件幸运的事。

其次,我要感谢我的父母,他们对我在生活,学习等方面的关心无微不至。在生活层面,他们十分关心我的衣食住行,时常也给我带来家的味道以缓解思乡之情。在学习和个人层面,他们也总对我论文写作,实习经历等关心有加。在我迷茫或是遇到挫折时,他们也总是给我坚定的支持与鼓励。

而后,我要感谢史妍,李超,姚晓婉,秦永菲,刘宇玮,王季欣,郭欣宇,朱黎彬等同学们,与他们一同进步的日子弥足珍贵。我也要特别感谢章寅同学,他在学术上带我入门,帮助我少走了很多弯路,也对我的成长帮助良多。此外,我也要感谢在之江实验室实习期间的领导陈红阳老师,以及遇到的包括吕劲松,陈虎,安丰,张世铭,王浩森等优秀的同事们。他们带我进一步接触学术前沿,提升阅历与能力。让我在实习期间收获不少温暖和成长。自信人生二百年,会当水击三千里!希望各位同学,同事在未来的工作生活中都能前程锦绣,万事顺遂。

最后,我要感谢张楚同学。你在我前行路上给予我陪伴,助我走的更远,更好,也在我遇到挫折时给予我鼓励和肯定,帮我看清前路的方向,也在我获得一些成果时,为我由衷高兴,为我注入继续成长的信心。

志之所趋,无远弗届,穷山距海,不能限也。希望我在未来的学习,工作和生活中,既能够脚踏实地的推进研究和工作进程,有条不紊的过好每一天的生活,又能天马行空的提出创新的想法,为平凡的前行道路上点亮漫天星光。

志不求易者成,事不避难者进,与诸君共勉。

\newpage