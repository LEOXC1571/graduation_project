\chapter{结论与展望}
\label{chap:conclusion}

\section{结论}
本文提出了一种创新的三维药物分子设计框架GFMDiff,该模型基于时下最优秀的生成模型扩散模型,通过对样本加噪声再去噪的方式生成真实有效的样本。本文的GFMDiff框架的去噪内核DTN中设计了针对原子特征和原子对特征学习的双轨Transformer架构。该网络能够完整的利用原子对距离和三元组角度几何信息,促进准确的特征学习并辅助模型训练和构象生成。同时由于该领域的方法在将扩散模型引入图生成的研究中严重依赖预设规则间接的生成边,这时常导致模型优化的方向与任务的评价指标不一致。与先前的方法不同,本文引入了GFLoss损失函数项在每个采样步中积极干预化学键的形成,这有效促进模型生成稳定的化学键和准确的分子构象。

为了验证GFMDiff在药物分子设计中的性能,本文在两个数据集,三个基准测试中开展了全面的实验,发现GFMDiff在普通药物分子设计,大分子设计,和设计具备特定属性的药物分子等任务中都显著优于该领域时下最优的模型。同时,为了验证DTN在特征学习中对分子几何信息的学习能力,本文在两个数据集的分子性质预测任务中开展全面的实验,证明了DTN是时下最优秀的分子性质学习方法之一。

经过实验检验,本文提出的GFMDiff是时下最优的创新三维药物分子设计框架,具备生成有效,稳定的药物分子的能力,进一步应用能够极大促进药物设计与开发。

\section{不足与展望}
尽管本文的GFMDiff已经是基于扩散的药物分子设计,甚至是所有药物分子设计领域表现最好的模型,但现有基于扩散模型的药物分子设计仍存在一些限制与不足。由于扩散过程天然更适合被用作连续变量空间中的样本生成,这与分子几何中的原子坐标不谋而合,但与图的离散属性有所冲突。虽然一些研究致力于通过改变噪声的形式,使扩散过程更好适应离散图的生成。有些研究引入状态转移矩阵作为噪声的表现形式,但这种形式显式的对边的关系进行编码,这可能在生成大图时面临急剧增加的计算需求。而有些研究将边删失引入噪声,这种方式虽然更直观且符合直觉,但该法面临在采样时生成过于稠密图的挑战。至今仍没有一个可以被同时用于大图和小图生成的通行扩散模型范式。本文通过间接的方式在训练时干预边的形成,但这种方法仅在分子生成领域有效。若需要推广至材料科学等交叉领域时,模型需要具备更多的专业知识。同时,本文对三维几何信息的提取时全面的,但也存在一定的冗余,加上DTN去噪内核中的多头注意力机制,本文对计算资源的需求仍较大。面临QM9中的中型分子时,模型训练需要2张A100 40G GPU,但在更大的药物分子生成任务中,或是移植到蛋白质生成的任务中时,模型在落地前可能需要一定的简化或加速。

尽管面临上述限制与挑战,本文提出的GFMDiff在中型分子生成时仍具展现出无可争议的性能。未来,相关的可能研究方向包括:

(1)DTN去噪内核虽然性能十分优异,但被应用到大分子数据集上时,其对计算资源消耗较大。故开发出于DTN性能相近,但消耗计算资源更小的去噪内核用于分子学习,是一个可能的研究方向。

(2)现有用于分子生成的大分子基准数据集GEOM-Drugs虽然包含了丰富的构象,但在药物分子设计任务中所需的一些属性和稳定构象仍有待完善。

(3)由于本文的方法属于基础方法,若要推进具体应用的落地,可以直接将本框架移植到蛋白降解靶向嵌合体药物生成等更接近实际应用的领域。

(4)鉴于大语言模型(LLM)近来的飞速发展,将大语言模型融入分子生成是一个可能的研究方向。将语义信息融入分子生成可以更好帮助下游应用的使用者快速熟练使用相关应用。

鉴于AI算法,GPU计算集群等软件和硬件技术在21世纪20年代初出现的飞跃式发展,我们有理由相信,人类距离进一步将AI推广到智能制造,智能计算等领域,助力智慧农业,新兴材料,生物医药,天文等领域实现爆发式发展,仅有一步之遥。同时我们也有理由展望,AI技术的快速发展与广泛落地能够带动新一轮的技术革命,并更好的普惠大众。