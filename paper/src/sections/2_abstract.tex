% Previous part
\cleardoublepage{}
\pagenumbering{Roman}
\thispagestyle{previous}
\pagestyle{previous}
\setcounter{page}{1}

\clearpage
\chapternonum{中文摘要}
% \begin{center}
%     \textbf{\heiti \zihao{3} \Title }
% \end{center}
\begin{center}
    \textbf{ \zihao{3} \heiti 摘\quad 要}
\end{center}
\songti \zihao{-4}
\linespreadonehalf

去噪扩散模型在多个研究领域显示出巨大潜力。现有的基于扩散生成模型在全新的三维药物分子设计任务中面临两个主要挑战。由于分子中的大部分重原子能够通过单键与多个原子相连,使用成对距离来建模分子几何结构是不足够的。因此,第一个挑战是提出一个有效的神经网络作为去噪核,能够捕捉复杂的原子间关系并学习高质量的特征。此外,由于图的离散性质,当前主流的基于扩散的药物分子设计模型严重依赖预设的规则,并以间接的方式生成边。故第二个挑战是将分子生成的过程与扩散的学习过程相结合,有效地准确预测键的存在。

本文认为扩散过程中分子构象的迭代更新方式与分子动力学一致,故提出一种名为几何辅助的分子扩散(Geometric-facilitated Molecular Diffusion / GFMDiff)的新型药物分子设计框架。针对第一个挑战,我们引入了一种双轨Transformer网络(Dual-track Transformer Network / DTN),以全面挖掘全局空间关系并学习高质量的表示,从而提高对特征和几何结构的准确预测能力。至于第二个挑战,我们设计了几何辅助损失(Geometric-facilitated Loss / GFLoss),该损失在训练期间干预键的形成,而不是直接将边嵌入隐变量空间。

本文在多个基准数据集上开展了全面实验表明GFMDiff性能达到该领域时下最优的水平。同时在公开数据集上的实验证明,去噪内核DTN具备强大的分子性质预测能力。


\vspace{20pt}
\noindent \textbf{关键字:}扩散模型;分子生成;分子学习;图神经网络;几何神经网络

% 英文摘要
\clearpage
\addcontentsline{toc}{chapter}{英文摘要}
% \begin{center}
%     \zihao{3} \TitleEng
% \end{center}π
\begin{center}
    \textbf{\zihao{3} Abstract}
\end{center}
Denoising diffusion models have shown great potential in multiple research areas. Existing diffusion-based generative methods on {\itshape de novo} 3D molecule generation face two major challenge. Since majority heavy atoms in molecules allow connections to multiple atoms through single bonds, modeling molecule geometries using pair-wise distance is insufficient. Therefore, the first one involves proposing an effective neural network as the denoising kernel that is capable to capture complex interatomic relationships and learn high-quality features. Due to the discrete nature of graphs, mainstream diffusion-based methods for molecules heavily rely on predefined rules and generate edges in a indirect manner. The second one involves accommodating molecule generation to the learning process of diffusion and accurately predicting the existence of bonds effectively. 

In our research, we view the iterative way of updating molecule conformations in diffusion process is consistent with molecular dynamics and introduce a novel molecule generation method named Geometric-Facilitated Molecular Diffusion (GFMDiff). For the first challenge, we introduce a Dual-track Transformer Network (DTN) to fully excevate global spatial relationships and learn high quality representations which contribute to accurate predictions of features and geometries. As for the second challenge, we design Geometric-facilitated Loss (GFLoss) which intervenes the formation of bonds during the training period, instead of directly embedding edges into the latent space. 

Comprehensive experiments on current benchmarks demonstrate the superiority of GFMDiff on molecule generation tasks and the outstanding capability of DTN on molecule property prediction tasks.

\vspace{20pt}
\noindent \textbf{Keywords: Diffusion models; Molecule generation; Molecular learning; Graph neural networks; Geometry neural networks}
\clearpage

% \cleardoublepage
% \addcontentsline{toc}{chapter}{目录}
\phantomsection
% \markloc
\tableofcontents

\begingroup
    \renewcommand*{\addvspace}[1]{}
    \cleardoublepage
    \phantomsection
    \marklof
    \listoffigures

    \cleardoublepage
    \phantomsection
    \marklot
    \listoftables
\endgroup

