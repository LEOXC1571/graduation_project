\mainmatter
\fancyfoot[EC,OC]{\hspace*{1 em}\thepage{}\hspace*{1 em}}
\normalsize
\fancypagestyle{plain}{\pagestyle{fancy}}
%\chapter[引言]{引言}\fancyhead[C]{\xiaowuhao 浙江工商大学硕士学位论文} %上传到系统里面的论文电子最终版本不要出现页眉(就是每一页的最顶端不要再写浙江工商大学了)
\chapter[引言]{引言}\fancyhead[C]{\xiaowuhao} %上传到系统里面的论文电子最终版本不要出现页眉(就是每一页的最顶端不要再写浙江工商大学了)
\section{选题背景与研究意义}
\subsection{选题背景}
除图像、视频,音频与自然语言处理等领域,AI技术的快速发展也带动相关交叉学科的发展。AI4Science近年在计算生物、计算化学、材料设计、计算天文,计算育种等都有广泛应用,相关AI技术的应用能够大幅加速相关科学研究进展。在计算制药领域,近年来相关AI技术在药物性质预测,生成,开发,实验等领域的运用不仅加速相关研究的进展,也能够降低相关研究的研发成本。

基于AI的生成模型近十年来也被广泛研究,他们包括变分自编码器(Variational autoencoders / VAEs)\cite{vae_kingma_13},生成对抗模型(Generative adversarial networks / GANs)\cite{gan_goodfellow_14},流形模型(Normalizing flows / NFs)\cite{nice_dinh_15,density_dinh_17},自回归(Autoregressive models / ARs)\cite{ar_oord_16}与扩散模型(Diffusion)\cite{deepunsupervised_dickstein_15,generative_song_19}等。相关方法在图像,文字等方面也有了许多成功应用。

深度学习在分子化学领域近年来也有着成功的应用。以分子学习为例,化学分子常以简化分子线性输入规范字符串(Simplified molecular-input line-entry system / SMILES)\cite{smiles_weinberger_88}存储,每一个分子式对应一个SMILES字符串。随着早期深度学习方法,如卷积神经网络(Convolutional neural networks / CNNs)\cite{cnnsmiles_hirohara_18}和循环神经网络(Recurrent neural networks / RNNs)\cite{rnnsmiles_bjerrum_17,practicalmodel_liu_19,deeppurpose_huang_20}的发展,一些研究试图运用这些深度学习算法,对以字符串形式存在的分子式进行学习,以获得预测特定原子或是分子整体的性质的能力。随着图神经网络(Graph neural networks / GNNs)\cite{semisupervised_kipf_17,inductive_hamilton_17,howpowerful_xu_18}的出现,其对非结构化数据的建模能力和对节点间拓扑关系学习的能力被证明十分优异。分子作为自然界中天然存在的图结构,原子和键对应着图中的节点和边,这为分子学习提供了新的思路与方法。从最早的图卷积神经网络开始,相关研究者致力于提出新的图学习算法,以提升对分子图学习的性能。随着相关化学模拟技术的发展,让三维分子建模成为可能。这也在拓扑结构信息以外,提供了更丰富的几何构型信息,这也驱动着相关研究拓展至几何图神经网络上。分子三维构象允许研究者对分子进行更准确的研究,同时也推动更多复杂任务的出现,包括分子生成,Ligand生成,Protacs生成,药物亲和力预测,蛋白质预测等等。

\subsection{研究意义}
人工智能在智能计算相关研究中开始扮演越来越重要的角色,相关模型在药物发现、药物属性预测等应用中已经展现出良好的性能和极大的潜力。由于深度学习技术应用具备为药物研发的多阶段降本增效的潜力,在2022年,AI制药赛道相关企业融资总金额达百亿美元。在这一赛道竞逐的有国内互联网巨头如百度百图生科、华为EIHealth、腾讯云深智药,及初创企业晶泰科技,剂泰医药,星药科技等。相关成果已经展现出深度学习在该领域的强大性能和广阔前景。

\section{文献综述}
\subsection{基于深度学习的分子学习}
分子最早被表示为简化分子线性输入规范字符串(SMILES)\cite{smiles_weinberger_88},随着早期深度学习模型卷积神经网络(CNN)和循环神经网络(RNN)的发展,相关模型利用CNNs和RNNs对分子性质做出学习。Hirohara等\cite{cnnsmiles_hirohara_18}的研究提出使用CNN对分子级别的特征和分子基团性质进行有效学习。由于RNNs在早期自然语言护理任务上有良好表现,Bjerrum\cite{rnnsmiles_bjerrum_17}提出LSTM-QSAR模型用于学习分子性质。

伴随图神经网络(GNNs)的发展,由于分子的形式天然的属于图结构,一些研究开始使用GNNs进行分子学习。CGCNN\cite{cgcnn_xie_18}将图卷积神经网络引入到分子性质学习,用以模拟并替代复杂的DFT计算。Xiong等\cite{attentivefp_xiong_20}提出Attentive FP,一个结合注意力机制的图神经网络实现分子性质的有效学习与预测。GraSeq\cite{graseq_guo_20}提出在运用图神经网络学习拓扑结构同时,利用双向LSTM模型对SMILES分子式进行学习,通过两个通道联合预测分子性质。伴随相关技术的发展,研究者可以不再拘泥于原子拓扑结构,进而实现对分子三维结构的建模与学习。鉴于某一分子对应大量的同分异构体,而不同构象对应属性不尽相同,因此对三维构象的有效学习是十分必要的。EGNN\cite{egnn_satorras_21}在流行的图网络基础上,保留最基本的几何信息,即原子间距离,其简单的设计也成为了一些。在分子预训练框架GEM\cite{gem_fang_22}中,研究者提出GeoGNN图网络,将键长视作原子节点图的边特征,又对原子键构图,并将键角作为原子键图的边特征,通过在两个网络上的信息传递实现分子局部空间几何性质学习。Sch\"{u}tt等\cite{schnet_schutt_17}在继承GNNs的信息传递范式的同时,将原子间距离用径向基函数(Radial basis funciton / RBF)建模后融入边特征,使算法对三维几何信息的有效学习的同时保证了等变性要求。SphereNet\cite{spherenet_liu_22}提出了基于球坐标系的信息传递范式SMP。通过一系列参考原子或键的规则,SMP在保证对原子对距离,键角和键扭转角这三个空间几何信息完整提取的同时,避免计算复杂度的爆炸式增长。ComENet\cite{comenet_wang_22}在ShpereNet的基础上,简化了空间几何信息的提取范式,在保证利用完整空间信息的前提下,以损失部分精度为代价,大幅度提升计算速度。

伴随着Transformer\cite{transformer_vaswani_17}相关研究在图像与文本领域的兴起,相关研究\cite{smilestrans_honda_19,smilesbert_wang_19,chemberta_chithrananda_20}也利用Transformer对SMILES分子式进行学习。随着GTN\cite{gtn_yun_19}将Transformer引入图学习,越来越多的研究也将多头注意力机制用于分子图学习领域。大规模分子图预训练框架GROVER\cite{grover_rong_20}中,分子学习内核运用了GTransformer,同时学习分子中的原子与键的节点嵌入(embedding)或边嵌入。在分子预训练框架MPG\cite{mpg_li_21}中提出的图学习内核MolGNet放弃了对边嵌入的学习,仅利用多头注意力学习节点嵌入,结果证明了该图学习算法的有效性。

\subsection{基于深度生成模型的分子设计}
自深度学习研究兴起以来,深度生成模型一直是研究者重点研究的对象。作画、翻译、对话,渲染等应用能够直接服务于广大用户。主流的深度生成模型有变分自编码器(Variational autoencoders / VAEs)\cite{vae_kingma_13},生成对抗模型(Generative adversarial networks / GANs)\cite{gan_goodfellow_14},流形模型(Normalizing flows / NFs)\cite{nice_dinh_15,density_dinh_17},自回归(Autoregressive models / ARs)\cite{ar_oord_16}与扩散模型(Diffusion)\cite{deepunsupervised_dickstein_15,generative_song_19}等。

与图学习的演进过程相似,基于深度生成模型的分子设计也经历了从二维图结构到三维几何构象的演进。主流分子设计任务具体又可以被细分为:分子生成,分子优化,构象生成,蛋白质配体生成,蛋白质生成,蛋白降解靶向嵌合体生成。

分子生成的任务就是根据给定分子数据,使模型具备凭空生成全新且有效的药物分子图或三维结构。考虑到复杂药物分子主要由官能团等子结构组成,JT-VAE\cite{jtvae_jin_18}基于VAE生成树结构骨架,而后利用树结构骨架逐步生成分子图结构。GraphVAE\cite{graphvae_simonovsky_18}是早期的基于VAE的图生成研究,为避免离散化结构的线性表示的相关障碍,使其中解码器直接输出预设的最大概率的全连接图。基于NF模型,MoFlow\cite{moflow_zang_20}将隐式表征逐步映射到条件流过程中,模型首先生成连接原子的键,随后通过图条件流生成原子,并最终组成有效的分子图。

现有的基于扩散的分子生成模型较少,此领域最早的研究为EDM\cite{edm_hoogeboom_22},其基于EGNN\cite{egnn_satorras_21}的去噪过程内核,通过生成点云,再根据预设定的化学性质生成连接边的键。MDM\cite{mdm_huang_22}在此基础上,提出了全新的去噪内核,通过两个SchNet\cite{schnet_schutt_17}分别对局部节点和全局节点特征进行学习。由于这种扩散模型更擅长在连续样本空间上的学习,故主流方法并不直接预测分子图中边的存在性,针对这一问题,DiGress\cite{digress_vignac_22}和MiDi\cite{midi_vignac_23}提出将图邻接矩阵引入扩散过程,并相应将马尔可夫状态转移矩阵引入噪声序列而非传统的高斯噪声。GCDM\cite{gcdm_morehead_23}针对对空间几何信息提取不足的问题,引入ColfNet\cite{colfnet_du_22}相似的空间几何学习范式,实现了对几何信息的充分学习。

\subsection{扩散模型}
在图学习领域中,一个图可以被表示为一个元组 $\mathcal{G} = (\mathcal{V}, \mathcal{E})$,其中包含了一个节点集合 $\mathcal{V}$ 和一个边集合 $\mathcal{E}$。在训练阶段,模型学习图在扩散过程中的概率分布,并被用于后续的采样阶段,以迭代地方式生成新的图。基于扩散的生成模型因其在多个领域的生成任务中表现出色而受到广泛关注,例如计算机视觉\cite{blendeddiffusion_avrahami_22,cascadeddiff_ho_22,gradforshapegen_cai_20,sgmpointcloud_luo_21},自然语言处理\cite{struccddpm_austin_21,argmaxflow_hoogeboom_21,stepunrolled_savinov_22},以及其他各种跨学科任务等\cite{cdvae_xie_22,housediffusion_shabani_23,nap_lei_23}。

人体骨架的表示形式与分子类似,都是表示为被边连接起的点云。不同之处在于生成人体骨架动作时,不需要对边的存在性进行预测。基于扩散模型,MoFusion\cite{mofusion_dabral_22}在人体动作生成中使用U-Net\cite{unet_ronneberger_15}作为扩散模型中去噪核心的架构。MoDi\cite{modi_raab_22}利用结构感知神经滤波器和3D卷积,实现对每个关节的精确控制。

除了在分子生成和人体动作生成上的应用外,许多研究工作致力于将扩散过程用于其他形式的图数据生成。SaGess\cite{sagess_limnios_23}通过引入广义分治框架来增强DiGress\cite{digress_vignac_22}。对于2D图生成,EDP-GNN\cite{edpgnn_niu_20}将Score SDEs结合到离散图邻接矩阵生成中。在此基础上,GraphGDP\cite{graphgdp_huang_22}提出了名为Position-enhanced图得分网络的去噪核心,以实现对位置信息的进一步利用。GSDM\cite{gsdm_luo_22}不仅仅在边属性上进行采样,还利用了节点特征和图谱空间上的Score SDE高效地生成图,这在通用数据集和分子数据集上得到了验证。受VAEs启发,NVDiff\cite{nvdiff_chen_22}采用VGAE结构,首先对图在隐变量空间上的特征进行采样,然后将其解码为节点或边特征。SLD\cite{sld_yang_23}以类似的方式进行图生成任务。GraphARM\cite{ardiff_kong_23}通过引入节点吸收扩散过程,将自回归模型与扩散模型结合起来。EDGE \cite{edge_chen_23}将扩散模型引入大图生成,扩散过程中逐渐移除边,直到图为空。该模型还通过仅关注部分节点来避免生成过多的边。专为双曲图设计的HGDM\cite{hgdm_wen_23}在提取双曲嵌入的复杂几何特征方面表现出优良的性能。



\section{创新点}
在本文中,我们提出了用于3D分子生成的“Geometric-facilitated Molecular Diffusion”(GFMDiff)框架。与先前的方法主要基于原子对距离学习原子特征不同,本文成功地将三元几何信息与原子对距离有效地结合到分子学习中。本领域大多数研究在生成式直接生成点云,并根据预设规则完成3D图结构的搭建。然而这种方法存在两个主要问题。首先,间接的图生成方式导致样本的稳定性和有效性下降。其次,传统的图卷积不足以区分局部和全局信息。为了解决第一个约束,本文设计了一个精巧的损失函数“Geometric-facilitated Loss”(GFLoss),在训练阶段主动引导键的形成。至于第二个约束,我们引入了“Dual-track Transformer Network”(DTN),这是一个基于全局Transformer的神经网络,以促进全面的几何学习和局部特征学习。

\section{基本框架}
本文可分为四章,每章的主要内容如下:

第一章为引言,本文在该部分介绍了分子生成的研究背景及意义,相关研究领域的研究现状及本文的主要创新点。

第二章中,本文在该部分介绍了分子图与扩散模型等预备知识。

第三章中,本文在该部分介绍了本文提出的GFMDiff框架。

第四章中,本文在该部分介绍了本文的GFMDiff模型分子生成任务中的表现及其中去噪核心DTN在分子学习任务上的表现。

第五章中,本文在该部分总结了本文研究的内容,并对未来发展做出展望。