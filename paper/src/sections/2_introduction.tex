\mainmatter
\fancyfoot[EC,OC]{\hspace*{1 em}\thepage{}\hspace*{1 em}}
\normalsize
\fancypagestyle{plain}{\pagestyle{fancy}}
%\chapter[引言]{引言}\fancyhead[C]{\xiaowuhao 浙江工商大学硕士学位论文} %上传到系统里面的论文电子最终版本不要出现页眉(就是每一页的最顶端不要再写浙江工商大学了)
\chapter[引言]{引言}\fancyhead[C]{\xiaowuhao} %上传到系统里面的论文电子最终版本不要出现页眉(就是每一页的最顶端不要再写浙江工商大学了)
\section{选题背景与研究意义}
\subsection{选题背景}

近年来基于深度学习的生成模型也在多领域有成功应用,例如AI在图画、语音、视频、对话等应用上的优秀表现引发了社会对人工智能新一轮广泛关注与热烈讨论。

在智能计算的计算医药相关研究中,深度学习模型在药物发现、药物属性预测等应用中已经展现出良好的性能和极大的潜力。人工智能技术应用能够为药物研发的多个阶段降本增效。过去的2022年,AI制药赛道相关融资总金额达百亿美元。国内互联网巨头如百度百图生科、华为EIHealth、腾讯云深智药,及初创企业晶泰科技,剂泰医药,星药科技等,相关成果已经展现出深度学习在该领域的强大性能和广阔前景。

\subsection{研究意义}

\section{文献综述}

\subsection{生成模型}

\subsection{分子学习}

\subsection{分子生成}
现有的基于扩散的分子生成模型有且只有EDM\cite{edm_hoogeboom_22}和MDM\cite{mdm_huang_22}。
 
\section{创新点}

\section{基本框架}