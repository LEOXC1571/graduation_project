\cleardoublepage
{
\chapternonum{附录}

\appendixsecmajornumbering

% \section{负对数似然估计}

% \begin{algorithm}[H]
%     \caption{Log-likelihood estimator for EDMs}
%     \label{alg:likelihood_edm}
%     \begin{algorithmic}
%     \STATE {\bfseries Input:} Data point $x$, neural network $\phi$
%     \STATE Sample $t \sim \mathcal{U}(1, \ldots, T)$, $\varepsilon_t \sim \mathcal{N}(\mathbf{0}, I)$, subtract center of gravity from $\varepsilon^{(x)}_t$ in $\varepsilon_t = [\varepsilon^{(x)}_t, ~\varepsilon^{(h)}_t$]
%     \STATE $z_t = \alpha_t [x, h] + \sigma_t \varepsilon_t$
%     \STATE $\mathcal{L}_t = \frac{1}{2}(1 - \mathrm{SNR}(t-1) / \mathrm{SNR}(t)) ||\varepsilon_t - \phi(z_t, t)||^2$
%     \STATE Sample $\varepsilon_0 \sim \mathcal{N}(\mathbf{0}, I)$, subtract center of gravity from $\varepsilon^{(x)}_0$ in $\varepsilon_0 = [\varepsilon^{(x)}_0, \varepsilon^{(h)}_0]$
%     \STATE $z_0 = \alpha_0 [x, h] + \sigma_0 \varepsilon_0$
%     \STATE $\mathcal{L}_0 = \mathcal{L}_0^{(x)} + \mathcal{L}_0^{(h)} = -\frac{1}{2} ||\varepsilon - \phi(z_0, 0)||^2 - \log Z + \log p(h | z_0^{(h)})$
%     \STATE $\mathcal{L}_{\text{base}} = -\mathrm{KL}(q(z_T | x, h) | p(z_T)) = -\mathrm{KL}(\mathcal{N}_{xh}(\alpha_T [x, h], \sigma_T^2 I) | \mathcal{N}_{xh}(\mathbf{0}, I))$
%     \STATE Return $\hat{\mathcal{L}} = T \cdot \mathcal{L}_t + \mathcal{L}_0 + \mathcal{L}_{\text{base}}$
%     \end{algorithmic}
% \end{algorithm}

% \begin{algorithm}[H]
%     \caption{新方法1}%{新方法1(modified EM algorithm 1, MEM1)}
%     \label{alg:New method 1}
%     \begin{algorithmic} %Display per line number
%     %   \Require 数据集$\{x_1, x_2, \cdots, x_n\}$, 成分数$K$;
%     %   \Ensure 参数$G$的估计值;
%       \State 参数初始值$G^{(0)}$;
%     %   \Repeat
%         % \State 计算$\tilde{\pi}_{ik}^{(m+1)}=\frac{f(x_i;\theta_k^{(m)})}{\sum_{j=1}^{K}f(x_i;\theta_j^{(m)})}$;
%         % \State 计算$\tilde{E}[z_{ik}|x_i;G^{(m)}]=\frac{\tilde{\pi}_{ik}^{(m+1)}f(x_{i}; \theta_{k}^{(m)})}{\sum_{j=1}^{K}\tilde{\pi}_{ij}^{(m+1)}f(x_{i}; \theta_{j}^{(m)})}$;
%         % \State 计算$G^{(m+1)}=\mathop{\arg\max}_{G} Q(G;G^{(m)})$,其中
%         % \begin{equation}
%         % \label{equ:Q function in new method 1}
%         %   Q(G;G^{(m)})=\sum_{i=1}^{n} \sum_{k=1}^{K} \tilde{E}[z_{ik}|x_i;G^{(m)}] [ \log\pi_{k} + \log\{f(x_{i}; \theta_{k})\} ];
%         % \end{equation}
%         % LaTeX: For the non numbered interline formula, why we use \[ … \] but $$ … $$?  see: https://www.zhihu.com/question/27589739/answer/37237684
%         %$\bm{\hat{\theta}}^{(m)}=\mathop{\arg\max}_{\bm{\theta}}\prod_{i=1}^{n}\{\sum_{j=1}^{K}\pi_{ij}^{(m-1)}f(x_i;\theta_j)\}$;
%         %\State compute $\pi_{ik}^{(m)}=\frac{f(x_i;\theta_k^{(m)})}{\sum_{j=1}^{K}f(x_i;\theta_j^{(m)})}$;
%         %\State compute $\pi_{k}^{(m)}=\sum_{i=1}^{n}\pi_{ik}^{(m)}/n$;
%     %   \Until{收敛}.
%     \end{algorithmic}
%   \end{algorithm}

\section{化学键长}
本文在生成药物分子中的化学键时,依据计算化学领域常见的原子键长作为参考。表~\ref{tab:single_bond_typ_dist}~,表~\ref{tab:double_bond_typ_dist}~和表~\ref{tab:tri_bond_typ_dist}~列出了出现在GEOM-QM9和GEOM-Drugs中原子类型间所有可能的化学键的典型键长。表中数值代表对应化学键的典型长度(皮米),而横线代表该两种原子间不可能存在稳定的对应化学键连接,原子间距离短于典型键长则可以被认为被对应的键连接。在实际计算中,对单键,双键和三键典型键长分别有10,5和3皮米的冗余。例如,对于两个距离136皮米的碳原子,他们之间距离虽然大于134皮米,小于154皮米,但鉴于136皮米小于134+5皮米,则认为这两个碳原子由双键连接。

\begin{table}[!h]
    \footnotesize
    \centering
    \caption{典型单键键长}
    \label{tab:single_bond_typ_dist}
    \begin{tabular}{l | r r r r r r r r r r r r r r r r r}
    \toprule
    & H & C & O & N & P & S & F & Si & Cl & Br & I & B & As \\ \midrule
    H & 74 & 109 & 96 & 101 & 144 & 134 & 92 & 148 & 127 & 141 & 161 & 119 & 152 \\
    C & 109 & 154 & 143 & 147 & 184 & 182 & 135 & 185 & 177 & 194 & 214 & - & - \\
    O & 96 & 143 & 148 & 140 & 163 & 151 & 142 & 163 & 164 & 172 & 194 & - & - \\
    N & 101 & 147 & 140 & 145 & 177 & 168 & 136 & - & 175 & 214 & 222 & - & - \\
    P & 144 & 184 & 163 & 177 & 221 & 210 & 156 & - & 203 & 222 & - & - & - \\
    S & 134 & 182 & 151 & 168 & 210 & 204 & 158 & 200 & 207 & 225 & 234 & - & - \\
    F & 92 & 135 & 142 & 136 & 156 & 158 & 142 & 160 & 166 & 178 & 187 & - & - \\
    Si & 148 & 185 & 163 & - & - & 200 & 160 & 233 & 202 & 215 & 243 & - & - \\
    Cl & 127 & 177 & 164 & 175 & 203 & 207 & 166 & 202 & 199 & 214 & - & 175 & - \\
    Br & 141 & 194 & 172 & 214 & 222 & 225 & 178 & 215 & 214 & 228 & - & - & - \\
    I & 161 & 214 & 194 & 222 & - & 234 & 187 & 243 & - & - & 266 & - & - \\
    B & 119 & - & - & - & - & - & - & - & 175 & - & - & - & - \\
    As & 152 & - & - & - & - & - & - & - & - & - & - & - & - \\
    \bottomrule
    \end{tabular}
\end{table}
    
\begin{table}[H]
    \centering
    \begin{minipage}[t]{.5\textwidth}
    \begin{table}[H]
    \footnotesize
        \centering
        \caption{典型双键键长}
        \label{tab:double_bond_typ_dist}
        \begin{tabular}{l | r r r r r r r r r r r r r r r r r}
        \toprule
        & C & O & N & P & S \\ \midrule
        C & 134 & 120 & 129 & - & 160 \\
        O & 120 & 121 & 121 & 150 & - \\
        N & 129 & 121 & 125 & - & - \\
        P & - & 150 & - & - & 186 \\
        S & - & - & - & 186 & - \\
        \bottomrule
        \end{tabular}
    \end{table}
    \end{minipage}
    \begin{minipage}[t]{.4\textwidth}
    \begin{table}[H]
    \footnotesize
        \centering
        \caption{典型三键键长}
        \label{tab:tri_bond_typ_dist}
        \begin{tabular}{l | r r r r r r r r r r r r r r r r r}
        \toprule
        & C & O & N \\ \midrule
        C & 120 & 113 & 116 \\
        O & 113 & - & - \\
        N & 116 & - & 110 \\
        \bottomrule
        \end{tabular}
    \end{table}
    \end{minipage}
\end{table}

    % \begin{figure}[htb]
    %     \centering
    %     \includegraphics{example-image}
    %     \caption{附录中的图片}
    %     \label{fig:test-appendix}
    % \end{figure}
    % \clearpage
    
\section{超参数设置}
由于完整跑完一轮训练所需时间较长,本文未对超参数设置进行完整的实验。但经过初步实验,本文确定了实验的相关超参数。在QM9和Drugs数据集上的模型训练时,学习率设定为0.001,DTN层数设定为5层,特征维度设定为256,注意力头数设定为8,神经元遗忘率为0.1,独热编码原子类型标准化系数为0.25,原子序数标准化系数为0.1,化合价标准化系数为0.1。在QM9数据集上的模型训练中,扩散步数为500,每批样本数量设定为32,而在Drugs上的训练中,扩散步数为1000,每批样本数量为1,这一设定受制于GPU显存容量限制。为实现等效训练的效果,本文通过梯度累积的方式,实现了等效批样本数量64的训练效果。

\section{DTN代码示意}
\begin{lstlisting}[style=python]
    import torch
    import torch.nn as nn
    import numpy as np
    from .utils import (atom_pos_to_pair_dist, remove_mean_with_mask, 
                           local_geometry_calc, RBF_Emb)
    from .layers import InterBlock, DistEncoder
    
    
    class DualTrackTransformer(nn.Module):
        def __init__(self, model_config, data_config):
            super(DualTrackTransformer, self).__init__()
            self.num_layers = model_config['num_layers']
            self.emb_dim = model_config['emb_dim']
            self.hidden_dim = model_config['hidden_dim']
            self.num_heads = model_config['num_heads']
            self.dropout = model_config['dropout']
            self.pair_scale = model_config['pair_loss_scale']
            self.context_dim = (len(model_config['context_col']) * 
                                   int(model_config['context']))
            self.x_class, self.v_class = data_config['x_class'], data_config['v_class']
            self.include_an = model_config['include_an']
            self.include_vl = model_config['include_vl']
            self.node_nf = data_config['x_class'] + self.include_an + self.include_vl
            self.act = nn.ReLU()
            self.edge_emb = DistEncoder(self.emb_dim)
            self.angle_emb = RBF_Emb(int(self.emb_dim/2), (np.arange(0, np.pi, 0.1), 10))
    
            self.convs = nn.ModuleList()
            for i in range(self.num_layers):
                self.convs.append(InterBlock(
                    emb_dim=self.emb_dim,
                    hidden_dim=self.hidden_dim,
                    num_heads=self.num_heads,
                    dropout=self.dropout,
                    dataset_name=data_config['name']
                ))
    
            atom_feat = self.node_nf + self.context_dim + 1
            self.xh_embedding = nn.Linear(atom_feat, self.emb_dim)
            self.node_pair_fuse = nn.Linear(3 * self.emb_dim, self.emb_dim)
            self.edge_emb_fuse = nn.Linear(2 * self.emb_dim, self.emb_dim)
            self.trip_emb_fuse = nn.Linear(self.emb_dim, int(self.emb_dim/2))
            self.xh_embedding_out = nn.Sequential(
                nn.Linear(self.emb_dim, self.hidden_dim),
                self.act,
                nn.Linear(self.hidden_dim, self.emb_dim),
                self.act,
                nn.Linear(self.emb_dim, atom_feat, bias=False)
            )
    
        def forward(self, z_t, t, node_mask, pair_mask, context=None):
            raise NotImplementedError
    
        def _forward(self, z_t, t, node_mask, pair_mask, context):
            pos, xh = z_t[:, :, :3], z_t[:, :, 3:]
            batch_size, max_node = xh.shape[0], xh.shape[1]
            if np.prod(t.size()) == 1:
                xh_time = torch.empty_like(xh[:, :, 0:1]).fill_(t.item())
            else:
                xh_time = t.view(batch_size, 1).repeat(1, max_node).unsqueeze(-1)
            xh = torch.cat([xh, xh_time], dim=2) * node_mask
            if context is not None:
                xh = torch.cat([xh, context], dim=2)
            xh_emb = self.act(self.xh_embedding(xh))
            xh_emb = xh_emb * node_mask
    
            _, radial, coord_diff = atom_pos_to_pair_dist(pos)
            radial = (radial.unsqueeze(-1) * pair_mask).squeeze(-1)
            coord_diff = coord_diff * pair_mask
            pair_emb = self.edge_emb(radial)
            pair_emb = self.node_pair_fuse(torch.cat([xh_emb.unsqueeze(1) * pair_mask,
                                                            xh_emb.unsqueeze(2) * pair_mask, 
                                                            pair_emb], dim=3))
            angle = local_geometry_calc(pos, pair_mask)
            tri_emb = self.act(self.angle_emb(angle))

            for layer in self.convs:
                pair_emb = pair_emb * pair_mask
                xh_emb, pair_emb, pos = layer(xh_emb, pair_emb, pos, coord_diff, node_mask, 
                                                   pair_mask, batch_size,
                                                   max_node, tri_emb=tri_emb)
                _, radial, coord_diff = atom_pos_to_pair_dist(pos)
                angle = local_geometry_calc(pos, pair_mask)
                tri_emb = self.act(self.trip_emb_fuse(torch.cat([self.angle_emb(angle), 
                                                                        tri_emb], dim=4)))     
                pair_emb = self.act(self.edge_emb_fuse(torch.cat([self.edge_emb(radial), 
                                                                         pair_emb], dim=3)))
                pair_emb = self.node_pair_fuse(torch.cat([xh_emb.unsqueeze(1) * pair_mask, 
                                                                xh_emb.unsqueeze(2) * pair_mask, 
                                                                pair_emb], dim=3))
    
            xh_emb = self.xh_embedding_out(xh_emb) * node_mask
            xh_final = xh_emb[:, :, :self.node_nf]
            vel = (pos - z_t[:, :, :3])
            vel = remove_mean_with_mask(vel, node_mask)
    
            return torch.cat([vel, xh_final], dim=2)
\end{lstlisting}


\section{GFLoss代码示意}

\begin{lstlisting}[style=python]
    def calc_gfloss(eps_t, eps, pair_mask=None, alpha=None):
        pos_t, x_t, a_t, v_t = eps_t[:, :, :3], eps_t[:, :, 3:3+self.x_class], 
                                   eps_t[:, :, -2], eps_t[:, :, -1]
        pos, x, a, v = eps[:, :, :3], eps[:, :, 3:3+self.x_class], 
                         eps[:, :, -2], eps[:, :, -1]
        x_prob = softmax(x_t.detach(), dim=-1)
        dist = (pos_t.unsqueeze(1) - pos_t.unsqueeze(2)).norm(dim=-1)
        pair_prob = (x_prob.unsqueeze(1).unsqueeze(-2) * x_prob.unsqueeze(2).unsqueeze(-1) *
                       self.norm_values[0])
        dist = dist.unsqueeze(-1).unsqueeze(-1).unsqueeze(-1).expand(x.shape[0], x.shape[1],
                x.shape[1], x.shape[2], x.shape[2], 3)
        dist_mar = (dist - self.bond.to(dist.device)) * self.bond_mask.to(dist.device)
        is_bond = (dist_mar.min(-1)[0] < 0).long() * pair_mask.unsqueeze(-1)
        bond_prob = pair_prob * is_bond
        valency = bond_prob.sum(-1).sum(-2).sum(-1) / self.norm_values[2]
        error = (((valency - v_t) ** 2) * alpha).sum(-1) / eps.shape[1]
        gfloss = self.loss_weight['degree'] * error
        return gfloss
\end{lstlisting}


% \section{难度可控的多项选择题生成实例}
}