\documentclass[a4paper,12pt,openany,oneside,utf-8]{ctexbook}
\usepackage{amssymb,amsmath}
\usepackage{subcaption}
\usepackage[amsmath,thmmarks]{ntheorem}
\usepackage{fancyhdr}
\usepackage{graphicx}
\usepackage{titletoc}
\usepackage{epsfig,picins,picinpar}
\usepackage{setspace}
\usepackage{geometry}
\geometry{left=3.3cm,right=2.8cm,top=2.5cm,bottom=2.2cm,}
\usepackage{mathtools}
\usepackage[super,square,comma,sort&compress]{natbib}
\usepackage{multirow}
\usepackage{dsfont}
\usepackage{booktabs}
\usepackage{epstopdf}
\usepackage{ccmap}
\bibliographystyle{gbt7714-2005}
\DeclareMathOperator*{\argmax}{arg\,max}
\usepackage{bm}
\usepackage{makecell}
\usepackage{booktabs}
\usepackage{threeparttable}
\usepackage{algorithm}
\usepackage{algorithmicx}
\usepackage{algpseudocode}
\usepackage{titlesec}
\usepackage{indentfirst}
\usepackage{caption}
\floatname{algorithm}{Algorithm}
\renewcommand{\algorithmicrequire}{\textbf{Input:}}
\renewcommand{\algorithmicensure}{\textbf{Output:}}
\usepackage{paralist}


\allowdisplaybreaks[4]
\renewcommand{\baselinestretch}{1.5}
%\renewcommand{\chaptername}{{Chapter\thechapter}}
\renewcommand{\chaptername}{第{\thechapter}章}
\renewcommand\bibname{参考文献}
\renewcommand\contentsname{目录}
%\renewcommand{\sectionname}{{\thechapter}.\arabic{section}}

\renewcommand {\thetable} {\thechapter{}-\arabic{table}} 
\renewcommand {\thefigure} {\thechapter{}-\arabic{figure}}

% 字体大小设置
\newcommand{\chuhao}{\fontsize{48pt}{\baselineskip}\selectfont}
\newcommand{\xiaochuhao}{\fontsize{36pt}{\baselineskip}\selectfont}
\newcommand{\yihao}{\fontsize{28pt}{\baselineskip}\selectfont}
\newcommand{\xiaoyihao}{\fontsize{25pt}{\baselineskip}\selectfont}
\newcommand{\erhao}{\fontsize{21pt}{\baselineskip}\selectfont}
\newcommand{\xiaoerhao}{\fontsize{17pt}{\baselineskip}\selectfont}
\newcommand{\sanhao}{\fontsize{15.75pt}{\baselineskip}\selectfont}
\newcommand{\xiaosanhao}{\fontsize{15pt}{\baselineskip}\selectfont}
\newcommand{\sihao}{\fontsize{13pt}{\baselineskip}\selectfont}
\newcommand{\xiaosihao}{\fontsize{12pt}{\baselineskip}\selectfont}
\newcommand{\wuhao}{\fontsize{10.5pt}{\baselineskip}\selectfont}
\newcommand{\xiaowuhao}{\fontsize{9pt}{\baselineskip}\selectfont}
\newcommand{\liuhao}{\fontsize{7.875pt}{\baselineskip}\selectfont}
\newcommand{\qihao}{\fontsize{5.25pt}{\baselineskip}\selectfont}%\newcommand\kaishu{\CJKfamily{kai}}
%这样在文档中就可以随心所欲地改变字体了, 比如上图中开始部分就是用
% {\yihao 1770年}{\sanhao 法国人狄道} 来实现的

\CTEXsetup[name={第,章}, number=\arabic{chapter}]{chapter} %将目录和正文的"第一章  XXXX"的输出格式改为"第1章 XXXX" see: https://blog.csdn.net/jk123vip/article/details/57163786 and http://tieba.baidu.com/p/3631630597
\titleformat{\chapter}{\centering\sanhao\bfseries}{第\,\thechapter\,章}{1em}{}
\titlespacing*{\chapter} {0pt}{-10.5pt}{12pt}

\titleformat{\section}{\centering\xiaosanhao\bfseries}{\thesection}{1em}{}
\titlespacing*{\section} {0pt}{12pt}{12pt}

\titleformat{\subsection}{\sihao\bfseries}{\thesubsection}{1em}{}
\titlespacing*{\subsection} {0pt}{0pt}{5.25pt}

\DeclareCaptionFont{fivehao}{\fontsize{10.5pt}{11pt}\selectfont #1} 
\pagestyle{fancy}%
%\renewcommand{\headrulewidth}{0pt}       %把页眉线的宽度设为零, 即去掉页眉线
\renewcommand{\footrulewidth}{0pt}
\addtolength{\headheight}{0\baselineskip}
\addtolength{\headwidth}{0\marginparsep}
\addtolength{\headwidth}{0\marginparwidth}
\setlength{\headsep}{5mm}
\fancyhf{}


\begin{document}
\theoremstyle{plain} \theoremseparator{}
\theoremindent0cm\theoremnumbering{arabic} \theoremsymbol{}% 以上是latex 的默认设置

%\vskip 26pt
%\renewcommand\refname{\Large\bf References}

%%%%%%%%%%%%%%%%%%%%%%%%%%%%%%%%%%%%

\fancypagestyle{plain}{%
\fancyhead{} 
% clear all header fields
%\fancyhead[CE,CO]{\xiaowuhao{浙江工商大学硕士学位论文}}} %上传到系统里面的论文电子最终版本不要出现页眉(就是每一页的最顶端不要再写浙江工商大学了)
\fancyhead[CE,CO]{\xiaowuhao{}}} 
%上传到系统里面的论文电子最终版本不要出现页眉(就是每一页的最顶端不要再写浙江工商大学了)
\begin{titlepage}
\fancypagestyle{plain}{\pagestyle{fancy}}
%\fancyhead[C]{\xiaowuhao 浙江工商大学硕士学位论文} % 上传到系统里面的论文电子最终版本不要出现页眉(就是每一页的最顶端不要再写浙江工商大学了)
\fancyhead[C]{\xiaowuhao} % 上传到系统里面的论文电子最终版本不要出现页眉(就是每一页的最顶端不要再写浙江工商大学了)
%%%%%%%%%%%%%%%%%%%%%%%%%%%%%%%%%%%%%%%%%%%%%%%%%%%%%%%%%%%%%%%%%%%%%%%%%%%% 完整
\vskip 4mm
\xiaowuhao 密级:公开 \hspace{9cm}中图分类号:O212.1

\vskip 6mm

\begin{figure}[htbp]
\centering
\includegraphics[width=140mm,height=22mm]{figures/zjgsu.jpg}
\end{figure}

\vskip 10mm

\begin{spacing}{1.0}
\begin{center}
\chuhao\textbf{硕士学位论文}
\end{center}

\vskip 20mm
\begin{center}
\hspace{0.01mm}\sanhao\textbf{论文题目:\underline{基于扩散模型的三维药物分子设计框架}}
\end{center}
\end{spacing}
\vspace{24mm}

\begin{center}
\textbf{\kaishu\sanhao 作者姓名:\underline{\quad \quad \quad \quad  徐璨 \quad \quad \quad \quad}}
\end{center}


\begin{center}
\textbf{\kaishu\sanhao 学科专业:\underline{\quad \quad \quad \ \ 统计学 \ \ \quad \quad \quad}}
\end{center}


\begin{center}
\textbf{\kaishu\sanhao 研究方向:\underline{\quad \quad \ \ \ 数理统计 \ \ \ \ \quad \quad}}
\end{center}

\begin{center}
\textbf{\kaishu\sanhao 指导教师:\underline{\quad \quad \quad \ \  王伟刚 \ \ \quad \quad \quad  }}
\end{center}

\

\

\


\begin{center}
\kaishu\sanhao 提交日期:2023 年12月
\end{center}
\end{titlepage}

\frontmatter
\pagenumbering{Roman}
\cfoot{\thepage}
\newpage
\begin{center}
\heiti\sanhao{基于扩散模型的三维药物分子设计框架}
\end{center}
\vskip 10 mm
\begin{center}
\heiti\sanhao{摘\quad 要}
\end{center}
\vskip 7 mm
\begin{spacing}{1.8}
{\sihao 近年来基于深度学习的生成模型也在多领域有成功应用,例如AI在图画、语音、视频、对话等应用上的优秀表现引发了社会对人工智能新一轮广泛关注与热烈讨论。}
\end{spacing}
\begin{spacing}{1.8}
{\sihao 在智能计算的计算医药相关研究中,深度学习模型在药物发现、药物属性预测等应用中已经展现出良好的性能和极大的潜力。人工智能技术应用能够为药物研发的多个阶段降本增效。过去的2022年,AI制药赛道相关融资总金额达百亿美元。国内互联网巨头如百度百图生科、华为EIHealth、腾讯云深智药,及初创企业晶泰科技,剂泰医药,星药科技等,相关成果已经展现出深度学习在该领域的强大性能和广阔前景。}
\end{spacing}
\begin{spacing}{1.8}
{\sihao 近来大火的生成模型被广泛应用于智能计算领域,人工智能算法有望根据人类的要求生成理想的结果,帮助提升药物发现与设计的效率与质量。在计算化学的相关研究中,深度学习模型的成功落地能够推动制药企业减少湿实验成本,助力靶点确认、药物发现、分子生成、化学反应设计、化合物筛选、临床试验、风险评估等多阶段。}
\end{spacing} 
\begin{spacing}{1.8}
{\sihao 本文聚焦于深度学习算法在三维药物分子发现这一主题,意在利用时下最优的生成模型算法,创新性设计出更符合三位药物理化性质的算法,提升全新药物分子设计的性能与效率。}
\end{spacing}
\vskip 16 mm
\noindent {\sihao 关键词: 扩散模型;分子生成;几何神经网络}


%%%%%%%%%%%%%%%%%%%%%%%%%%%%%%%%%%%%%%%%%%%%%%%%%%%%%%%%%%%%%%%%%%%%%%%%%%%%%%%%%%%%%%%%%%%%%%%%%%%%%%%%
                                                                                                       %
\newpage                                                                                               %
\begin{spacing}{1.1}
\begin{center}
\heiti\xiaosanhao{A DIFFUSION-BASED 3D MOLECULE GENERATIVE FRAMEWORK}
\end{center}
\end{spacing}
\vskip 14mm
\begin{center}
\heiti\xiaosanhao{ABSTRACT}
\end{center}
\vskip 7mm

AAAAAAAAAAAAA

\vskip 14 mm
\noindent{KEYWORDS: Diffusion model; Molecule generation; Geometry neural network}

\mainmatter
\fancyfoot[EC,OC]{\hspace*{1 em}\thepage{}\hspace*{1 em}}
\normalsize
\fancypagestyle{plain}{\pagestyle{fancy}}
%\chapter[引言]{引言}\fancyhead[C]{\xiaowuhao 浙江工商大学硕士学位论文} %上传到系统里面的论文电子最终版本不要出现页眉(就是每一页的最顶端不要再写浙江工商大学了)
\chapter[引言]{引言}\fancyhead[C]{\xiaowuhao} %上传到系统里面的论文电子最终版本不要出现页眉(就是每一页的最顶端不要再写浙江工商大学了)
\section{选题背景与研究意义}
\subsection{选题背景}

近年来基于深度学习的生成模型也在多领域有成功应用,例如AI在图画、语音、视频、对话等应用上的优秀表现引发了社会对人工智能新一轮广泛关注与热烈讨论。

在智能计算的计算医药相关研究中,深度学习模型在药物发现、药物属性预测等应用中已经展现出良好的性能和极大的潜力。人工智能技术应用能够为药物研发的多个阶段降本增效。过去的2022年,AI制药赛道相关融资总金额达百亿美元。国内互联网巨头如百度百图生科、华为EIHealth、腾讯云深智药,及初创企业晶泰科技,剂泰医药,星药科技等,相关成果已经展现出深度学习在该领域的强大性能和广阔前景。

\subsection{研究意义}

\section{文献综述}

\subsection{生成模型}

\subsection{分子学习}

\subsection{分子生成}
现有的基于扩散的分子生成模型有且只有EDM\cite{edm_hoogeboom_22}和MDM\cite{mdm_huang_22}。
 
\section{创新点}

\section{基本框架}

\chapter[分子图学习与基于扩散模型的图生成]{分子图学习与基于扩散模型的图生成}
\label{chap:diffusion-based_molgen}

\section{分子图学习}
在图学习中通常有两类监督任务:节点分类/回归和图分类/回归。对于图学习,一个分子可以被抽象为一个图 $\mathcal{G} = (\mathcal{V}, \mathcal{E})$,其中 $|\mathcal{V}| = n$表示节点(原子)的集合,$|\mathcal{E}| = m$ 表示分子中的边(化学键)的集合。我们用$e_i$表示节点 $i$的特征,用$e_{ij}$表示边$(i, j)$的特征。

在节点分类或回归任务中,每个节点$i$都有一个标签或目标$y_i$。任务目标是通过学习,预测未见节点的标签。这个任务可被应用于许多应用,比如识别分子中的功能团或预测各个原子的性质。

此外,在图分类或回归任务中,给定一组图$\{ \mathcal{G}_1, \mathcal{G}_2, ..., \mathcal{G}_N \}$和对应的标签或目标 $\{ y_1, ..., y_N \}$。此时任务目标是根据图的结构和节点边的特征,预测给定图的标签或目标。这个任务被广泛应用于化合物分类或基于结构预测分子性质等。

在这两类任务中,目标是利用监督学习技术训练模型,有效地捕捉图的结构与相关标签/目标之间的关系。现有的图学习方法主要有基于图神经网络的模型和基于Transformer架构的模型。

图神经网络(Graph Neural Networks,简称GNNs),近来在知识图谱、社交网络和药物发现等各个领域引起了广泛的关注。GNNs的核心操作在于将图中节点或边之间的特征进行传递(也称为邻居聚合)。消息传递操作通过聚合节点$i$的邻居节点和边的隐式特征来迭代更新节点$i$自身的隐式特征$e_i$。一般来说,消息传递过程包含多轮迭代,每轮迭代可以被视为对更远距离邻居的消息聚合。假设有$L$轮迭代,第$l$轮迭代会将目标节点的l跳邻居特征注入目标节点的隐式特征。第$l$轮迭代中,消息的传递与聚合可被表示为:
\begin{eqnarray}
    &m_j^{(l)} = {\rm MSG}^{(l)}(e_j^{(l-1)}), \ j \in \mathcal{N}_i, & \\
    &e_i^{(l)} = {\rm AGG}^{(l)}(\{ m_j^{(l)}, \ j \in \mathcal{N}_i \}, \ m_i^{(l)} ), &
\end{eqnarray}
m(l,k) = AGGREGATE(l)({(h(l,k−1)v, h(l,k)v) | v与节点v相邻},其中m(l,k)v是聚合后的消息,σ(·)是某个激活函数。我们约定h(l,0)v := h(l−1,Kl−1)v。选择AGGREGATE(l)(·)的方式有几种流行的方法,如平均值、最大池化和图注意力机制。对于一次消息传递,其中有一个可训练参数层(即AGGREGATE(l)(·)、W(l)和b(l)内部的参数)。这些参数在迭代l中的Kl个跳数之间是共享的。经过L次消息传递后,最后一次迭代中最后一个跳数的隐藏状态被用作节点的嵌入,即h(L,KL)v,其中v ∈ V。最后,应用READOUT操作来获取图级别的表示。

model: gpt-3.5-t

\section{基于扩散模型的图生成}

\chapter[几何促进的3D分子图生成]{几何促进的3D分子图生成}
\label{chap:gfmdiff}

在本节中 本文将着重介绍3D分子生成的模型框架

\begin{figure}[h]
    \centering
    \includegraphics[width=\linewidth]{figures/overview_gfmdiff.png}
    \caption{Overview of GFMDiff}
    \label{fig:gfmdiff}
  \end{figure}


\section{双轨Transfoermer网路}

\section{几何促进的损失函数}

\section{扩散及去噪过程}

\section{目标函数}

\chapter[实验结果及分析]{实验结果及分析}
\label{chap:experiment}

\section{分子生成}

\subsection{实验设置}
在本节中,我们报告了GFMDiff在两个主流数据集GEOM-QM9 \cite{qm9_ramakrishnan_14}和GEOM-Drugs \cite{drugs_axelrod_22}上的性能。结果表明,我们的方法在多个方面显著优于一些最先进的模型。

为了进行全面的比较,我们在两个分子生成的基准数据集上进行了实验:GEOM-QM9 \cite{qm9_ramakrishnan_14}和GEOM-Drugs \cite{drugs_axelrod_22}。GEOM-QM9数据集包含超过130K个分子及其对应的构象,其中平均每个分子有18个原子,包括氢原子。GEOM-Drugs是一个规模较大的数据集,包括的分子数量和平均分子数量都较多。它包含超过450K个分子和37M个构象,其中平均分子大小为44。

\subsection{在GEOM-QM9数据集上的全新三维分子生成}


\begin{table}[h]
    \centering
    \caption{Performance comparison on QM9}
    \label{tab:exp_qm9}
    \begin{tabular}{llllll}
    \hline
    Method & NLL$\downarrow$ & \makecell[l]{Atom\\Stable\\(\%) $\uparrow$} & \makecell[l]{Mol\\Stable\\(\%) $\uparrow$} & \makecell[l]{Valid\\(\%) $\uparrow$} & \makecell[l]{Unique$\cdot$Valid\\(\%) $\uparrow$} \\
    \hline
    E-NF & -59.7 & 85.0 & 4.9 & 40.2 & 39.4 \\
    G-SchNet & N/A & 95.7 & 68.1 & 85.5 & 80.3 \\
    EDM & -110.7$\pm$1.5 & 98.7$\pm$0.1 & 82.0$\pm$0.4 & 91.9$\pm$0.5 & 90.7$\pm$0.6 \\
    Bridge & N/A & 98.7$\pm$0.1 & 81.8$\pm$0.2 & N/A & N/A \\
    Bridge+Force & N/A & \underline{98.8}$\pm$0.1 & 84.6$\pm$0.3 & N/A & N/A \\
    GCDM & \textbf{-171.0}$\pm$0.2 & 98.7$\pm$0.0 & 85.7$\pm$0.4 & 94.8$\pm$0.2 & 93.3$\pm$0.0 \\
    \hline
    \makecell[l]{GFMDiff\\w/o tri} & -123.1$\pm$0.4 & 98.7$\pm$0.1 & 85.9$\pm$0.2 & 94.9$\pm$0.2 & 94.2$\pm$0.2 \\
    \makecell[l]{GFMDiff\\w/o GFLoss} & -127.5$\pm$0.4 & 98.7$\pm$0.0 & \underline{86.5}$\pm$0.1 & \underline{95.2}$\pm$0.0 & \underline{94.5}$\pm$0.0 \\
    GFMDiff & \underline{-128.0}$\pm$0.2 & \textbf{98.9}$\pm$0.0 & \textbf{87.7}$\pm$0.2 & \textbf{96.3}$\pm$0.3 & \textbf{95.1}$\pm$0.2 \\
    \hline
    Data &  & 99.0 & 95.2 & 97.7 & 97.7 \\
    \hline
    \end{tabular}
\end{table}

\subsection{在GEOM-QM9数据集上的条件三维分子生成}

\begin{table}[h]
    \centering
    \caption{Performance comparison on QM9}
    \label{tab:exp_qm9_condition}
    \begin{tabular}{lllllll}
    \hline
    \makecell[l]{Task\\Units} & \makecell[l]{$\alpha$\\${\rm Bohr^3}$} & \makecell[l]{$\Delta \varepsilon$\\${\rm meV}$} & \makecell[l]{$\varepsilon_{{\rm HOMO}}$\\${\rm meV}$} & \makecell[l]{$\varepsilon_{{\rm LUMO}}$\\${\rm meV}$} & \makecell[l]{$\mu$\\${\rm D}$} & \makecell[l]{$C_v$\\$\frac{{\rm cal}}{{\rm mol}}{\rm K}$} \\
    \hline
    Naive (Upper-bound) & 9.01 & 1470 & 645 & 1457 & 1.616 & 6.857 \\
    \# Atom & 3.86 & 866 & 426 & 813 & 1.053 & 1.971 \\
    EDM & 2.76 & 655 & 356 & 584 & 1.111 & 1.101 \\
    GCDM & \underline{1.97} & \underline{602} & \underline{344} & \underline{479} & \underline{0.844} & \underline{0.689} \\
    GFMDiff & \textbf{1.74} & \textbf{558} & \textbf{321} & \textbf{430} & \textbf{0.728} & \textbf{0.593} \\
    QM9 (Lower-bound) & 0.10 & 64 & 39 & 36 & 0.043 & 0.040 \\
    \hline
    \end{tabular}
\end{table}


\subsection{在GEOM-Drugs数据集上的三维分子生成}

\begin{table}[h]
    \centering
    \caption{Performance comparison on Drugs}
    \label{tab:exp_drugs}
    \begin{tabular}{llll}
    \hline
    Type & Method & Atom Stable (\%) $\uparrow$ & Mol Stable (\%) $\uparrow$ \\
    \hline
    Normalizing flow & E-NF & 75.0 & 0 \\
    \multirow{2}{*}{DDPM}
    & EDM  & 81.3 & 0.0 \\
    & Bridge & 81.0$\pm$0.7 & 0.0 \\
    & Bridge+Force & 82.4$\pm$0.8 & 0.0 \\
    & GCDM & 86.4$\pm$0.2 & 3.7$\pm$0.3 \\
    \hline
    Ours & GFMDiff & \textbf{86.5}$\pm$0.2 & \textbf{3.9}$\pm$0.2 \\
    \hline
    Data &  & 86.5 & 2.8 \\
    \hline
    \end{tabular}
\end{table}

\section{分子性质预测}

\subsection{实验设置}

\subsection{在GEOM-QM9数据集上的分子性质预测}

\begin{table}[h]
    \centering
    \caption{Performance comparison on QM9}
    \label{tab:exp_qm9_reg}
    \begin{tabular}{lllllll}
    \hline
    \makecell[l]{Task\\Units} & \makecell[l]{$\alpha$\\${\rm Bohr^3}$} & \makecell[l]{$\Delta \varepsilon$\\${\rm meV}$} & \makecell[l]{$\varepsilon_{{\rm HOMO}}$\\${\rm meV}$} & \makecell[l]{$\varepsilon_{{\rm LUMO}}$\\${\rm meV}$} & \makecell[l]{$\mu$\\${\rm D}$} & \makecell[l]{$C_v$\\$\frac{{\rm cal}}{{\rm mol}}{\rm K}$} \\
    \hline
    Naive (Upper-bound) & 9.01 & 1470 & 645 & 1457 & 1.616 & 6.857 \\
    \# Atom & 3.86 & 866 & 426 & 813 & 1.053 & 1.971 \\
    EDM & 2.76 & 655 & 356 & 584 & 1.111 & 1.101 \\
    GCDM & 1.97 & 602 & 344 & 479 & 0.844 & 0.689 \\
    GeoDiff &  &  &  &  &  & \\
    QM9 (Lower-bound) & 0.10 & 64 & 39 & 36 & 0.043 & 0.040 \\
    \hline
    \end{tabular}
\end{table}

\subsection{在OC20数据集上的分子性质预测}


\chapter[结论与展望]{结论与展望}
\label{chap:conclusion}

\input{sections/7_contributions.tex}

\input{sections/8_acknowledgement.tex}

\end{document}