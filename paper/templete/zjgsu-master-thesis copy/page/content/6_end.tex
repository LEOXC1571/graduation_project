本章将简要汇总本文的研究工作,并根据当前主流研究趋势对标签感知推荐系统进行未来研究展望。
\section{工作总结}
本文的研究工作主要集中在利用社交标签信息实现用户个性化物品推荐,旨在从社交标签数据中学习用户和物品的嵌入式表达,并结合个性化推荐算法实现 Top-K 推荐。本研究关注的核心问题在于如何充分利用内涵丰富且形式简洁的标签信息,以获得更好的用户和物品表达,进而提高推荐效果。同时,本文旨在合理利用图神经网络和对比学习来提高标签感知推荐的性能。

当前的研究存在三个主要问题:1)现有的推荐系统涉及到的交互类型繁多,如何高效组织数据,并进一步建模数据中的多种交互类型是极具挑战的;2)现有应用图神经网络的模型仅使用原始神经网络的设定,而没有根据标签感知推荐系统进行进一步调整;3)现有模型所采用的优化机制往往忽略了图神经网络自身的特点,盲目地沿用传统推荐算法的优化目标进行模型优化。

为了解决这些问题,本文基于图神经网络和对比学习提出了两个标签感知推荐模型 LFGCF 和 TAGCL。LFGCF 模型将社交标签信息与用户和物品嵌入式表达相结合,通过利用轻量化图神经网络,使其更适应与社会标签数据。TAGCL 模型则采用了对比学习,以更好地去除推荐数据中常见的流行性偏差,并有效地提高标签感知推荐的性能。在公开数据集上,实验结果表明,本文提出的两个模型均优于现有的主流推荐算法,具有良好的推荐效果。具体来说,本文的研究工作主要包含以下几个方面:

(1) 本文旨在研究标签感知推荐系统,将其研究重点划分为数据与算法两个部分。在数据部分,本文分析过去文献对数据建模方式的问题,并针对社会标签数据提出了合理的建模方式。在算法部分,本文将标签感知推荐系统抽象为数学形式,为后续研究打下基础。

(2)本文提出基于图神经网络的轻量化标签感知推荐模型 LFGCF ,该模型针对推荐系统的特点去除了图卷积神经网络的特征变换和非线性激活组件,并采用加权和聚合函数进行消息传播,提高了模型推荐的精准度并减轻了训练难度。

(3) 本文还探讨了推荐系统中存在的流行度偏差现象,并提出了一种新的模型 TAGCL。该模型使用对比学习和知识图谱联合优化模型,在训练过程中对标签进行同时采样,有效地提高了模型推荐的精准度和公平性。

(4)本文通过设计一系列实验并与当前主流推荐算法模型进行对比,评估了 LFGCF 和 TAGCL 的性能。此外,为验证模型的有效性和实用性,本文还在一个真实运行的系统数据中验证了模型的有效性。

综合来看,本文为推荐系统中的标签感知和不公平问题提供了新的思路和解决方案,并提出了可行的轻量化标签感知推荐模型 LFGCF 和新型的不公平感知推荐模型 TAGCL。

\section{未来研究展望}
对于标签感知推荐系统的进一步研究,本文认为有以下几点值得关注和探索:

(1) 使用超图来建模社会标注行为。超图(hypergraph)是一种广义的图结构,在信息科学和生命科学等领域得到了广泛应用。与传统的图不同,超图的一条边可以连接任意数量的顶点。考虑到标签可以被多个用户使用并描述多个物品,因此使用超图来进一步定义社会标注数据具有非常大的潜力。

(2) 深入研究图神经网络的更深层次。图神经网络的深度可以反映节点信息表征的来源范围。然而,随着深度的增加,过拟合(over-fitting)、梯度消失(gradient vanishing)和过度平滑(over-smoothing)等问题变得越来越严重。在学术研究级别的数据上,使用较浅的图神经网络可以得到不错的效果。但在真实的工业场景中,小模型无法满足大规模数据的需求。因此,我们需要探索更深层次的图神经网络,并解决其面临的问题。

(3) 探索基于软最大损失函数的推荐算法优化方向。当前主流的推荐算法优化方向仍然是成对型损失函数。然而,该损失函数会引入归纳偏差,即对于用户没有交互过的物品,很多是因为推荐系统没有将其推荐给用户。因此,基于软最大损失函数的方法可以更有效地引入负样本,以减少这种偏差。

(4) 推荐系统的可解释性。对于用户来说,推荐系统的解释是非常重要的。如果用户无法理解推荐系统的决策过程,那么他们很可能会对推荐结果产生怀疑,从而降低推荐系统的使用率。因此,我们需要探索如何使推荐系统更加可解释。例如,可以使用可视化技术来展示推荐系统的决策过程,并提供解释。
