\section{系统实现}
为了验证本文提出的推荐模型在实际应用场景中的有效性,本小节针对工业级推荐系统的架构设计和开发实现了一个简单的原型系统,包括推荐系统框架介绍、需求分析与功能设计以及最终的展示效果。在推荐系统框架介绍阶段,本节将推荐系统的分解为前端展示、数据和推荐服务模块。在需求分析和功能设计阶段,本节分析标签感知推荐系统需要的关键组件。最终展示效果方面,本原型系统采用了用户界面友好、操作简单的交互设计,能够快速响应用户的请求并输出个性化推荐结果。通过实际应用测试,本原型系统表现出良好的性能和准确性,能够为用户提供更加优质的推荐服务。

\subsection{推荐系统框架}
现有的工业级大规模推荐系统架构一般由三部分组成:提供用户交互的前端展示模块、执行数据操作和存储的数据模块、执行具体推荐算法和策略的推荐服务模块。如图~\ref{fig:rs-systems}。接下来本小节讲简要地介绍各个模块的核心功能:

(1)\textbf{前端展示模块}

前端展示模块作为推荐系统中至关重要的一环,扮演着向用户提供推荐交互体验的核心角色。它由展示界面和日志系统两部分构成。前端展示界面是用户与推荐系统进行交互的主要界面,其核心功能是为用户推荐感兴趣的资源,并以列表、瀑布流的形式呈现推荐内容,以提供卓越的推荐服务。此外,前端展示界面还负责所有同用户交互的功能,例如搜索、筛选、排序、评分、评论等,以提升用户的满意度和体验。另一方面,日志系统作为前端展示模块的辅助系统,主要负责采集用户在界面上的所有动作,例如对某个资源的点击、评分、评论等,记录用户的行为数据,以供后续的数据模块和推荐服务模块使用。对于标签感知推荐系统而言,除了常见推荐系统所提供的功能外,还需要提供社会标注系统,以收集社会标注数据。社会标签系统允许用户为物品打上标签,从而为推荐系统提供额外的信息。

(2)\textbf{数据模块}

数据模块是推荐系统的重要组成部分之一,其主要任务是从用户历史行为日志中解析信息,生成可供推荐模型使用的数据流,其中包括数据的采集、处理和存储等功能。对于从前端展示模块采集的数据,需要先进行数据清洗,过滤掉冗余数据,并通过数据流拼接生成可供模型使用的数据块,并划分为训练数据和测试数据。近年来,Apache Storm + Flink流式数据处理框架因其部署灵活、伸缩性强、极致的流式处理性能等特点备受工业界青睐。采用Flink可以为各种存储系统(如Kafka和JDBC数据库系统)提供丰富的连接器,同时在线数据流会被周期性地保存到HDFS等离线介质中,以实现数据的持久化存储。Flink具有低延迟、高吞吐量、高容错性等特点,能够高效地进行数据处理,支持在数据流中进行实时计算和流式机器学习,为推荐系统的数据模块提供了强大的支持。本原型系统使用 SQLite 作为后端数据库,该数据库具有部署简单、迁移便利等优点,并且得到了大多数软件的支持。

(3)\textbf{推荐服务模块}

推荐服务模块是推荐系统的核心模块之一,负责执行实际的推荐策略。由于工业级数据规模较大且用户兴趣多样,现有的推荐服务通常分为召回和排序两个阶段。在召回阶段,利用多种策略根据用户的兴趣、行为习惯、当前环境和热点新闻等,进行多路召回,以召回足够数量的候选集,并对结果进行合并、去重、过滤等操作以确保推荐结果的多样性。由于推荐模型较为复杂,为了提供实时的交互响应,通常将排序分为粗排和精排两个步骤。在粗排阶段,通过浅层模型对候选集中的物品规模进行初筛,留下数量较少的物品进入精排,再通过较为复杂的深层精排模型执行最终的 Top-K 排序。排序之后,顶层的规则系统会对排序结果进行强插、打散、聚合等规则,以生成最优的展示列表,从而提供用户最佳的浏览体验。为了满足工业级数据规模的需求,推荐服务模块需要具备高效的计算和处理能力,同时保证推荐结果的准确性和多样性,这对于现有的推荐系统提出了较高的挑战。本原型系统简化了以上过程,只保留最后的个性化排序过程,为用户提供个性化排序列表。

\subsection{需求分析与功能设计}

本文的研究旨在探讨如何利用用户标注的标签数据实现更加精准的个性化推荐算法,而不是建立大规模推荐系统。鉴于此,本节设计了一个精简的原型系统,只选取了部分核心模块,避免了过度复杂的设计。具体来说,该原型系统的核心功能需求包括:标签数据的采集与处理、标签感知推荐模型的构建与训练、推荐结果的生成与展示等。

\subsubsection{前端展示模块}
为了实现基于标签感知推荐算法,本文选择以 Web 页面的形式作为前端展示模块来展示推荐结果。该原型系统的核心功能包括用户登陆、推荐列表展现、用户对资源打标签以及日志记录等部分。主要核心模块包括用户登陆、推荐列表展现、社会标注功能以及日志记录这几个部分。具体内容如下:

(1)在用户登陆方面,该原型系统采用公开数据集中的用户 id 作为用户名进行身份认证。这一设计简化了用户登陆流程,同时也可以实现与现有数据的结合,从而提高了系统的可用性和实用性。本系统也额外爬取了某电影网站数据,作为本系统的推荐对象。最后,本系统也支持新创建用户,新创建用户 id 将会在现有用户 id 上自增实现。

(2)在推荐列表展现方面,系统展示核心的 Top-K 推荐结果。推荐列表以图文列表的形式展示,包括被推荐物品的信息,并且展示该资源已有的部分标签。这一设计有助于用户理解推荐结果的原因,并提高了推荐的可解释性。

(3)用户打标签是系统的核心功能之一,该功能提供了为资源标注的功能,体现了用户对资源的看法和态度。

(4)志记录功能用于记录用户在界面的一切行为,主要记录用户对资源的标注行为,以生成新的数据实例。系统仅采集用户的标注记录作为正样本,后续工作将会将用户没有点击的资源作为负样本并记录,从而为模型提供更加准确的负样本数据。这些数据将为推荐算法的性能提高提供有力的支持。

\subsubsection{数据模块}
数据模块是推荐系统的重要组成部分,其核心功能在于记录并保存用户历史行为数据,生成数据流。由于本系统是离线计算的,不具备流式更新的功能,因此数据模块只需要选择较为简单的关系型数据库即可。数据库记录的信息主要包括资源信息、用户信息、标签信息、交互信息以及推荐信息。

(1)物品信息:物品信息包括资源的唯一标识符id以及相关信息,如名称、图片url等。物品的唯一标识符 id 将用于模型的输入特征,其余物品信息将用于推荐模型的前端展示。

(2)用户信息:用户信息包括用户的唯一标识符id和相关信息,如姓名、性别等。用户的唯一标识符 id 将用于模型的输入特征。其余用户信息将会被保存,为后续扩展作出准备。

(3)标签信息:标签信息包括标签的唯一标识符id和相关信息,如标签含义。标签的唯一标识符 id 将用于模型的输入特征。标签信息的记录将作为前端描述资源的内容。

(4)交互信息:交互信息包括标注列表信息,如用户id、物品id、标签id、时间戳等。这些信息将被用于生成用户的历史行为数据,从而更好地理解用户的兴趣和行为。

(5)推荐信息:推荐信息包括用户id和推荐的物品id列表。这些信息将被用于评估推荐模型的准确性,并对推荐结果进行后续分析和优化。通过对这些信息的记录和分析,可以更好地理解用户兴趣和需求,提高推荐系统的精度和效果。

\subsubsection{推荐服务模块}
推荐服务模块是本系统的核心模块,其主要任务是执行模型的离线训练和在线预测,以实现对用户的个性化推荐。本系统不需要实现工业界的召回/排序两阶段架构,而是将推荐服务简化为仅由本文第三章提出的模型作为最终的排序模型。推荐服务模块包含离线训练和在线预测两个部分。

(1)离线训练:离线训练部分提供一个可视化的配置界面,可以选择训练模型(如 LFGCF、TAGCL和其他基线模型)。并将模型中的超参数(如学习率)以可配置的形式展现,方便系统管理员对模型进行配置和训练。模型训练时,从数据模块中取出训练数据,传递到深度学习框架 PyTorch 进行离线模型训练。通过离线训练,模型可以学习到用户兴趣和行为特征,并通过排序算法对推荐列表进行排序。

(2)在线预测部分通过用户 id 查询用户的嵌入表征,从而提取用户特征,执行模型的预测算分服务,得到排序后的资源并返回前n个,并将推荐结果记录到数据库中。在线预测部分的关键是如何提取用户的交互数据,本文采用了基于用户标注历史记录的方法。对于已登录的用户,根据用户id查询数据库,获取推荐列表并在前端展示。由于推荐服务模块是基于离线计算的,因此可以大幅提高系统的运行效率,同时保证推荐的实时性。

\subsection{系统设计}
本小节将详细阐述原型系统的前端展示模块、数据模块和推荐服务模块的系统设计。

前端展示模块是系统的用户接口,主要由 web 页面和动态的推荐信息流构成。采用的开发技术为 HTML5+CSS+JavaScript,并选择了流行的 bootstrap 前端框架,以提供简洁的界面风格和良好的交互体验。用户登陆后,以用户 id 为参数请求后台数据库,查询推荐数据表,获得当前的推荐资源 id 列表。接着对资源 id 列表进行解析,并逐一查询资源数据表,获取资源的相关信息(如图片 url,歌手姓名,标注标签列表等),构造推荐数据流 response。最后,将推荐数据流返回给前端,在 web 界面中展示。当用户对资源进行标注时,会记录用户行为日志并传递给数据模块。

数据模块主要进行数据的记录和保存操作,核心功能是记录并保存用户历史行为数据,生成数据流。由于系统是离线计算的,不具备流式更新的功能,因此数据模块只需选择较为简单的关系型数据库即可。数据库中记录的信息主要包括资源信息、用户信息、标签信息、交互信息和推荐信息。其中,资源信息包括资源 id 和相关信息,用户信息包括用户 id 和相关信息,标签信息包括标签 id 和含义,交互信息包括标注列表信息(如用户 id、资源 id、标签 id、时间戳等),推荐信息包括用户 id 和推荐的资源 id 列表。

推荐服务模块是执行模型离线训练和在线预测的核心模块。系统的重心在于展示模型的推荐效果,因此将工业界的召回/排序两阶段架构简化为仅由本文第三章提出的模型作为最终的排序模型。推荐服务模块包含离线训练和在线预测两部分。离线训练提供一个可视化的配置界面,可以选择训练模型进行离线训练。模型训练时,从数据模块中取出训练数据,传递到深度学习框架 PyTorch 进行离线模型训练。在线预测阶段,对于每一个用户,通过用户 id 查询用户的标注历史记录,提取用户特征,执行模型的预测算分服务,得到排序后的资源并返回前 n 个,并生成推荐记录 传递给数据模块保存到数据库中。

\subsection{实现效果}
本小节将详细展示原型系统的实现效果。该系统包括模型后台管理和前端推荐展示界面两大部分内容。

模型后台管理主要为模型离线训练提供简洁的可视化操作界面,如图()所示。管理员可以在模型训练界面中选择数据源(训练数据)、训练模型(LFGCF或TAGCL)、PyTorch训练器版本,选择是否使用预训练模型,调节模型相关的训练参数,并配置训练平台硬件资源(如GPU、内存)。参数配置完成后,管理员可以提交模型训练任务。当模型训练任务完成后,系统会记录本次训练过程并保存离线训练模型,并将用户推荐表更新到数据库中。

前端推荐界面主要提供用户浏览资源的界面,展现个性化推荐效果。用户在登录界面输入用户名并登录后,系统将跳转到推荐页面,页面主体内容为推荐的资源列表(如图中推荐的歌手列表),并展示歌手被其他用户标注过的部分标签信息,从而为用户提供推荐解释依据。同时,用户可以点击每个推荐项中的“Label”按钮对该资源进行打标签,并生成新的标注记录。

由于本文的研究重心是探索标签感知推荐系统的算法模型,因此本节搭建的原型系统只包含推荐系统的核心功能。系统的不足之处在于没有在线训练的设计,模型不能流式更新,因此需要周期性重训。此外,推荐的结果仅由单个模型决定,没有召回/排序两阶段来提供良好的综合推荐性能。

\section{本章小结}
本章主要对提出的两个标签感知推荐模型 LFGCF 和 TAGCL 荆襄实验验证和性能对比分析。首先,本章介绍了实验所使用的数据集和数据预处理过程,并将数据集的统计信息与流行度偏差进行可视化。接着阐述具体的实验方法、实验平台和实验流程。在进行多次实验后,整理数据,并对实验结果进行深入分析。实验表明,与现有的通用推荐模型和标签感知推荐模型相比,本文提出的模型在多个评估指标上都有较为显著的性能提升。同时,为了深入理解模型,本文还进行了丰富的消融实验与超参数实验,并对比分析了提出的两个模型与其他模型的参数、推理复杂度。最后,本文设计并实现了一个原型系统,从工业级推荐系统架构的介绍、原型系统的需求、最终功能设计和展示效果展开介绍。